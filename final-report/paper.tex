\documentclass[12pt]{sigplanconf}

% The following \documentclass options may be useful:

% preprint      Remove this option only once the paper is in final form.
% 10pt          To set in 10-point type instead of 9-point.
% 11pt          To set in 11-point type instead of 9-point.
% authoryear    To obtain author/year citation style instead of numeric.

\usepackage{amsmath}
\usepackage{shortcut}
\usepackage{subcaption}
\usepackage{graphicx}
\usepackage{hyperref}

\newtheorem{definition}{Definition}[section]

\begin{document}

\special{papersize=8.5in,11in}
\setlength{\pdfpageheight}{\paperheight}
\setlength{\pdfpagewidth}{\paperwidth}

\titlebanner{banner above paper title}        % These are ignored unless
\preprintfooter{short description of paper}   % 'preprint' option specified.

\title{Title Text}
\subtitle{Subtitle Text, if any}

\authorinfo{Jaeho Kim}
{KAIST \\
    Daejeon, South Korea
}
{oojahooo@kaist.ac.kr}
\authorinfo{Tae Eun Kim}
{KAIST \\
    Daejeon, South Korea
}
{}
\authorinfo{Seunghyeon Jeong}
{KAIST \\
    Daejeon, South Korea
}
{}
\authorinfo{Kihwan Kim}
{KAIST \\
    Daejeon, South Korea
}
{}

\maketitle

\begin{abstract}
    This is the text of the abstract.
\end{abstract}

\section{Introduction}
Fuzzing is a technique that is widely applied in the real world with its simple intuition and excellent applicability.
For example, Google provides a large platform named OSS-Fuzz for open source software. Developers can upload their open
source project to this platform and get reports about bugs that founded by executing the fuzzer. Actually, OSS-Fuzz has
helped identify and fix over 8,900 vulnerabilities and 28,000 bugs across 850 projects~\footnote{https://github.com/google/oss-fuzz\#trophies}.

Greybox fuzzing is a generally used fuzzing technique that explores the program guided by code coverage. It mutates the
seed inputs in a way that increases code coverage. The term `Greybox' means that the fuzzer does not use whole information
of program's code, but it uses only code coverage information.

\begin{figure}[h]
    \cFormat
    \lstinputlisting[]{figure/ex1.c}
    \hrule width \hsize height .33pt
    \caption{Condition Example of a Simple Program that Takes String Input}
    \label{fig:string-example}
  \end{figure}

Because of the aforementioned characteristic of greybox fuzzer, it struggle with the absence of code coverage. Especially,
it is challenging for greybox fuzzer if there are conditions with no intermediate code coverage. Figure~\ref{fig:string-example}
shows the example of the condition with no intermediate code coverage. At the line 6, the input argument should be exactly
``HELLO'' in order to execute true branch and crash statement at line 8. So executing with input string ``HELL'' and
``HEAVEN'' respectively have same code coverage. It is fatal problem for mutator in greybox fuzzing. ``HELL'' is more
similar (i.e. less edit distance) with ``HELLO'' than ``HEAVEN'', so it can make ``HELLO'' by adding only one character,
but the fuzzer evaluates those two inputs are same effectiveness. Therefore the fuzzer makes more insignificant inputs,
and it takes too much time.

To overcome this problem, we present a new task for encoding the value coverage in string domain, using finite state machine,
and combine it with directed fuzzing technique. Directed fuzzing is a new technique of fuzzing that aims a specific location
of the program. So it is more efficient than greybox fuzzing when we want to find an input that causes program to crash
at specific location.

\section{Overview}
\subsection{Finite State Machine}
Finite state machine is a useful structured representation that represents a program's behavior. In general, a program is
a transition system. Every program has states and transition functions. So we can express a program defining states and
transition function. Actually, Depending on how the state and transition function are defined, an infinite number of states
may be required to express a program. But in a specific domain, we can represent it using finite number of states. So we
utilize the finie state machine for only representing conditional expression containing string library functions.

We define finite state machine like below for using our domain.

\begin{definition}
    $(Q, \Sigma, \delta, q_0, F)$ is a finite state machine where $Q$ is finite set of states, $\Sigma$ is a set of input
    symbol, especially ascii characters in our domain, $\delta : (Q \times \Sigma) \rightarrow Q$ is a partial function
    that takes current state and current input symbol, $q_0$ is initial state (i.e. state that input symbol is empty string),
    and $F$ is a set of accepted (final) states.
\end{definition}

\subsection{Examples and Methods}

\begin{figure}[h]
    \begin{subfigure}[t]{0.45\textwidth}
        \cFormat
        \lstinputlisting[]{figure/strcmp.c}
        \hrule width \hsize height .33pt
        \caption{Scenario that uses strcmp function}
        \label{fig:strcmp}
    \end{subfigure}
    \begin{subfigure}[t]{0.45\textwidth}
        \cFormat
        \lstinputlisting[]{figure/strstr.c}
        \hrule width \hsize height .33pt
        \caption{Scenario that uses strstr function}
        \label{fig:strstr}
    \end{subfigure}
    \begin{subfigure}[t]{0.45\textwidth}
        \cFormat
        \lstinputlisting[]{figure/atoi.c}
        \hrule width \hsize height .33pt
        \caption{Scenario that uses atoi function}
        \label{fig:atoi}
    \end{subfigure}
    \caption{Three scenarios that usually occurs in many programs}
    \label{fig:scenarios}
\end{figure}

We choose three scenarios that usually occurs in many programs that take string inputs, 1) using \verb|strcmp| function
in conditional expression, 2) using \verb|strstr| function in conditional expression, 3) using \verb|atoi| function in
conditional expression. Figure~\ref{fig:scenarios} is the simple code examples that contains our scenarios.

Each scenario makes greybox fuzzer struggle with the absence of code coverage. In Figure~\ref{fig:strcmp}, we have to find
exactly ``123456'' as an crashing input. In Figure~\ref{fig:strstr}, we should find the string that contains ``1234'' as
an crashing input. In Figure~\ref{fig:atoi}, we have to find the string that represents the integer 123456.

To make new coverage method for these scenarios, we use the finite state machine. Finite state machine can express the
sequence of transition from starting input symbol to final input symbol, and check whether the input string is accepted.

\begin{figure}[h]
    \begin{subfigure}[t]{0.45\textwidth}
        \includegraphics[width=\textwidth]{figure/strcmp.png}
        \caption{Contructed FSM for strcmp}
        \label{fig:fsm-strcmp}
    \end{subfigure}
    \begin{subfigure}[t]{0.45\textwidth}
        \includegraphics[width=\textwidth]{figure/strstr.png}
        \caption{Contructed FSM for strstr}
        \label{fig:fsm-strstr}
    \end{subfigure}
    % \begin{subfigure}[t]{0.45\textwidth}
    %     \includegraphics{figure/atoi.png}
    %     \caption{Contructed FSM for atoi}
    %     \label{fig:fsm-atoi}
    % \end{subfigure}
    \caption{Constructed finite state machines for each scenario}
    \label{fig:fsm} 
\end{figure}

So we construct the finite state machines for each scenario, and add it as new conditional statement these to original code.
Figure~\ref{fig:fsm} shows the FSMs constructed for each scenario. And we implement these FSMs in real program as adding
new code.

You can see the detailed implementation in our repo~\footnote{https://github.com/goodtaeeun/smAFL}.

\section{Evaluation}
We have an experiment for comparing the efficiency of our technique with traditional fuzzers. So our baselines are traditional
greybox fuzzer, AFL~\cite{aflfuzz}, and traditional directed fuzzer, AFLGO~\cite{bohme:ccs:2017}.

\subsection{Experimental Setup}

\subsection{Result}

\appendix
\section{Appendix Title}

This is the text of the appendix, if you need one.


\bibliographystyle{ACM-Reference-Format}
\bibliography{references}


\end{document}